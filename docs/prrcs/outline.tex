\documentclass{prrcs}

%====================================================

\title{%
    \vspace{-0.75cm}%
    Holistic Dashboard of Software\\Development Work-Time \& Practices%
}

\author{%
    \vspace{-0.5cm}%
    Inesh Bose \& Dr. Tim Storer (University of Glasgow)%
}

\date{%
    % \vspace{-0.25cm}%
    \footnotesize{Note: Please find acronyms at the end of the paper.}%
    \vspace{0.5cm}%
}

\begin{document}
\maketitle

%====================================================
\section{Introduction (Outline)}

Through the research project introduced in this proposal, we aim to understand and determine areas of improvement in Developer Experience (DX); this is because \textbf{humans are the most important aspect of software engineering \cite{martinAgileSoftwareDevelopment2003}}, and eliminating frustrations in DX can enable the delivery of high-quality products \cite{sommervilleSoftwareEngineering1992} with good User Experience (UX).

Humans interact with computers using interfaces, and while average end-users use high-level interfaces, developers provide lower levels of instructions to the computer. Most notably, the essential interface tool is a form of Integrated Development Environment (IDE) with the main element being the Text Editor. Some research and steps have been carried out to improve the IDE and implement better alerts, debuggers, syntax highlighting, and IntelliSense, but there are often gaps in projects and/or within teams causing failure in quality delivery.

%====================================================
\section{Background \& Rationale}

There are various reasons for subjective developer experience for an objective output. For example, choosing and knowledge of the right tools and building on well-laid foundations, especially within teams. Humans need to be given visual feedback on their interface \cite{liCategorisationVisualisationMethods2016}, so visualisation is important in the field of HCI. Git \cite{Git} has been very useful in showing changes through commits and diffs, and platforms such as GitHub \cite{GitHubLetBuild} \& Bitbucket \cite{BitbucketGitSolution} provide additional features such as automated pipelines to run tests on changes and alert the developers \cite{FeaturesGitHubActionsa,GitLabCICD,atlassianBitbucketPipelinesContinuous}. Services like Codacy \cite{DevOpsIntelligencePlatforma}, Code Climate \cite{DataDrivenEngineeringIntelligence}, Libraries.io \cite{LibrariesOpenSourcea} and Snyk \cite{SnykDeveloperSecurity2020a} specialise in different aspects of code such as quality\footnote{based on technical debts, complexities, etc.}, maintainability, dependencies and vulnerability. Dependabot \cite{Dependabot} alerted developers on GitHub and secured 18M projects in 2022 \cite{GlobalDeveloperCommunity}. So feedback \& awareness\footnote{Nielsen (1994) heuristics 1 and 9} is an essential aspect of DX, but there's a lack of an intuitive method of revealing their development habits, and this could be very useful in reminding developers about the state and the health of their projects. We present some particular areas related to DX and research in them.

\subsection*{Productivity}

% for example, Zoom during the COVID-19 pandemic required many bugfixes and features implemented in short iterations;
Organisations are insistent on faster deliveries of software, so they tend to measure productivity\footnote{using metrics like SLOC \cite{careyImpactCommunicationMode1997,barry1981software,conteSoftwareEngineeringMetrics1986,jonesProgrammingProductivity1985}} \cite{devanbuAnalyticalEmpiricalEvaluation1996}, but it is not linear \cite{trendowiczChapterFactorsInfluencing2009,abdel-hamidSlipperyPathProductivity1996,briand2002software,kemererEmpiricalValidationSoftware1987}. There has been research to understand developer productivity and its factors; \textcite{trendowiczChapterFactorsInfluencing2009} lists these based on over a hundred publications, eventually learning that \textbf{the first step of quantifying productivity management is selecting the right metrics}.

\subsection*{Learning}

Experience and knowledge of tools is a major factor to productivity \cite{gillProductivityImpactsSoftware1990,deomeloInterpretativeCaseStudies2013,maxwellBenchmarkingSoftwareDevelopmentProductivity2000,oliveiraSoftwareProjectManagers2016}. \textcite{cockburnCostsBenefitsPair2001,murphy-hillHowUsersDiscover2015} look into "social" learning of new tools for developers (like pair-programming). In many cases, there is a huge difference in developers knowing all the features that their IDE offers \cite{campbellDesigningRefactoringTools2008}. \textcite{begelNoviceSoftwareDevelopers2008} studies new recruits (novices) at a large organisation to understand how they progress and transition to experts, finding that onboarding processes are tailored for this purpose and involve effective methods such as mentoring and pair-programming.

\subsection*{Structure}

The scale within a team is very different. \textcite{basili1979investigation,manteiEffectProgrammingTeam1981} found relationships between the size \& organisation of a programming group and several software metrics revealing Controlled Decentralised teams to be effective over CPT \cite{mills1971chief,bakerChiefProgrammerTeam1972} and EPT \cite{weinbergPsychologyComputerProgramming1988}. Similarly, \textcite{sawyerSoftwareDevelopmentTeams2004, heTeamCognitionDevelopment2007} focus on the communication of information and determine that hybrids of Sequence, Group \& Network archetypes and frequent meetings (over emails) have positive effects on team cognition.

\subsection*{Experience}

Significant specialised research has not yet been conducted for DX and there are only limited formal papers available on it. As discussed in previous sections, there are many factors affecting the development process and experience, like productivity and mood \cite{WhyDoesProductivity,beechamMotivationSoftwareEngineering2008,graziotinFeelingsMatterCorrelation2015,amabileProgressPrincipleUsing2011,meyerSoftwareDevelopersPerceptions2014,mullerStuckFrustratedFlow2015,khanMoodsAffectProgrammers2011}. We need to provide developers appropriate means of measure for their interpretation and improvement of their DX.

%====================================================
\section{Suggested methodology}

This project plans to provide a plug-in extension for Visual Studio Code \cite{VisualStudioCode} text editor that presents a dashboard to the developers of a project to assess recent activities.

The implementation would make use of the Webview API \cite{WebviewAPI} to serve a SvelteKit \cite{SvelteCyberneticallyEnhanced} frontend that displays graphs generated from a modular core package. This is aimed to be similar to fitness apps such as Fitbit \cite{FitbitAppDashboarda} and Google Fit \cite{GoogleFita} where one can enable graphs for metrics of interest\footnote{hence the project name "Code Fitness" \cite{boseCodefitness2022}}\footnote{terms like "CodeFit" or "DevFit" may cause clashes}.

The default set of services' plugins for the dashboard to generate graphs would be for GitHub \cite{GitHubLetBuild} and WakaTime \cite{wakatimeWakaTimeDashboardsDevelopers} as they allow a good amount visualisation about developer activities (amount of time spent in a file for a commit, time spent researching \& debugging), especially if the two are integrated.

%====================================================
\section{Expected outcomes}

We hope to discover how the dashboard affects developers and their projects. Using surveys and open-ended interviews, we would see if the tool could pose to be useful, how developers and their teams engaged \& looked for customisation on it, and understand what metrics are interesting and useful to them if it empowers their development process. Along with that, analysis of the team members' activity data (with consent) could be helpful to further understand their behaviour and verification of any generated hypothesis.

%====================================================
\section*{Acronyms}

\begin{itemize}
    \setlength\itemsep{-0.5em}
    \item DX - Developer Experience
    \item UX - User Experience
    \item HCI - Human-Computer Interaction
    \item IDE - Integrated Development Environment
    \item CPT - Chief Programmer Team
    \item EPT - Ego-less Programming Team
\end{itemize}

\let\oldbibliography\thebibliography
\renewcommand{\thebibliography}[1]{\oldbibliography{#1}
\setlength{\itemsep}{-3pt}}

% \bibliographystyle{abbrv}
%\setstretch{0.8}
{
\scriptsize
\printbibliography[notkeyword=tech]
\printbibliography[keyword=tech,heading=subbibliography,title={Technologies}]
% \printbibliography[notcategory=cited,heading=subbibliography,title={Further Reading}]
}
\end{document}
