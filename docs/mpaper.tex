\documentclass{mpaper}

\usepackage[numbers]{natbib}
\usepackage[nameinlink]{cleveref}

% temp fix for subfigures
\usepackage[labelfont={bf},font={bf}]{caption}
\usepackage{subcaption}

\usepackage{subfiles}

\let\autoref=\Cref
\newcommand{\noteurl}[1]{\footnote{\url{#1}}}
\setlength{\abovecaptionskip}{10pt plus 1pt}

\graphicspath{{./images/}}
\newcommand{\optionalnote}[1]{}

\begin{document}

%==============================================================================
% METADATA

\title{%
    A Holistic Dashboard of Software%
    \vskip4pt%
    Development Work-Time \& Practices%
}
\author{Inesh Bose}
\matricnum{2504266}
\date{} % \date{April 14, 2023} % {\today}

\maketitle

%==============================================================================
% ABSTRACT
% According to Simon Peyton Jones, an abstract should address four key questions.
% First, what is the problem that this paper tackles?
% Second, why is this an interesting problem?
% Third, what is the solution this paper proposes?
% Finally, why is the proposed solution a good one?

\begin{abstract}
This project researches Developer Experience, task-effort estimations and the relationship between them; this is because failures \& frustrations in software project comes from the lack of awareness of code with bad DX (as it is given low regard) and inaccurate resource allocation as there are no tools for developers to reflect on their work-time \& practices. We provide a solution through the means of a centralised modular dashboard showing visualisations of their development activity as realistic breakdowns from the discrete events and objective metrics on their code repository. This was evaluated in realistic scenarios with 20 developers of various roles in different structures, as they work on their projects, to determine the usability and effect on their process, showing increased awareness of their practices, and realisation of establishing good DX in their codebase. % with a lot of scope for the future.

\vskip6pt \noindent
{\bf Concepts.} User experience; task estimations; version control

\vskip4pt \noindent
{\bf Keywords.} Developer experience; developer productivity % developer relations; web development
\end{abstract}

%=============================================================================

\section{Introduction}\label{sec:Introduction}
\subfile{chapters/01_introduction}

\section{Related Work}\label{sec:RelatedWork}
\subfile{chapters/02_relatedwork}

\section{Design}\label{sec:Design}
\subfile{chapters/03_design}

\section{Implementation}\label{sec:Implementation}
\subfile{chapters/04_implementation}

\section{Evaluation}\label{sec:Evaluation}
\subfile{chapters/05_evaluation}

\section{Conclusions}\label{sec:Conclusion}
\subfile{chapters/06_conclusion}

%=============================================================================
% BIBLIOGRAPHY

\vskip8pt \noindent
{\bf Acknowledgments.}
Thanks to everyone and every organisation, especially the JavaScript/Vue/Nuxt community, that has been a part of my Computing Science journey academically and professionally helping me enjoy creating software with no pain for users and developers equally. I'm grateful and appreciative of Dr. Tim Storer for supervising me on this project and being understanding from both research-perspective and engineering-perspective. Finally, I would also like to thank \textbf{you} for giving your valuable time to read through this paper - it means the world. Thank you.

\renewcommand{\bibsection}{\section*{REFERENCES}}
\setlength{\bibsep}{5pt}
\bibliographystyle{abbrvnat}
{%\footnotesize
\bibliography{%
    references/primary,%
    references/secondary,%
    references/tertiary,%
    references/technology,%
    references/repository,%
    references/example%
}
}

\end{document}
