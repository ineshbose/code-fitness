%=============================================================================
% CONCLUSIONS

\documentclass[../mpaper.tex]{subfiles}
\begin{document}

This paper has discussed the importance of Developer Experience in relation to productivity and its impact on the success of projects. We presented a new tool that exposes developers' activities carried out for completing tasks on their repository to provide more accurate and realistic measures of productivity. The initial evaluation of this tool was carried out with 20 developers of different demographics such as roles, disciplines and team structures; it reports that developers were able to gain more detail to each revision in a repository and realise the time spent on it, helping them improve their practices such as committing frequently and consistently. The usability score for our system (73.8) in such early stages shows promise and scope for the project, but not all developers were able to perceive the metrics, thus not seeking to use it in the current stage.

In regard to the limitations of our study, first, the time constraints included implementing our tool, recruiting participants and running longitudinal evaluations; this didn't allow much flexibility and left little time for each phase to either be rushed or cut down. Second, due to the tight schedule, it was convenient to recruit friends \& colleagues at the University of Glasgow for our experiments; this opened the potential for bias in the results as they may have provided diplomatic and socially desirable responses to the questions. Moreover, this sample of 20 participants was small and not representative of all developers as they were all novice graduate students at the university with relatively limited knowledge and experience in industrial software engineering. When the implementation was finished and ready to be evaluated, the developers had already crossed the research \& learning phase of their projects and were on the final iteration. Finally, third, the implemented tool, only developed in a few weeks, did not give a seamless experience to the users and had limitations in the data such as accuracy, proper detailing and graph issues (such as axis, toggles, and size).

Developer Experience is a new \& highly subjective field; any research in understanding this area will benefit software projects everywhere. More plugins and longer structured experiments for the system can provide more insight. The evaluation of our tool brought out useful feedback from the developers such as providing integration with other IDEs or repository wikis, alerting developers on new activities, and storing diary entries in a new tab. The idea of the dashboard was to help developers and their teams make accurate estimates for future tasks by basing their considerations on their experience \& metrics for the previous ones, but it was up to the developers to infer the data. As also noted by a participant's feedback, there is potential for the system to specialise in providing estimation facilities such as taking story-points as input and verifying that effort metrics correspond to it or predict otherwise using algorithms. Moreover, investigating work-time practices to generate visualisations has made us realise the large amount of context switching that developers require, affecting the mental load and ultimately the DX. Our dashboard integrated into VSCode because we mentioned the importance of having information central to the SCM \cite{edwardsSciitEmbeddingIssue2021}; but the implementation could help promote more information to the central dashboard. There is also scope for relating DX \& Productivity with emerging tools like ChatGPT and GitHub Copilot through the dashboard. Overall, this study has opened up new avenues for future research in the field of Developer Experience.

\end{document}
