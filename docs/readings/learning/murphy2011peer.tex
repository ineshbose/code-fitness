\section{Peer Interaction Effectively, yet Infrequently, Enables Programmers to Discover New Tools}

\subsection{Abstract}

Computer users rely on software tools to work effectively and efficiently, but it is difficult for users to be aware of all the tools that might be useful to them. While there are several potential technical solutions to this difficulty, we know little about social solutions, such as one user telling a peer about a tool. To explore these social solutions in one particular domain, we describe a series of interviews with 18 programmers in industry that explore how tool discovery takes place. These interviews provide a rich set of qualitative data that give us detailed insights into how programmers discover tools. One finding was that, while programmers believe that discovery from peers is effective, they actually discover tools from peers relatively infrequently. Another finding was that programmers can effectively discover tools from their peers both in a co-located and remote settings. We describe several implications of our findings, such as that discovery from peers can be enhanced by improving programmers’ ability to communicate openly and concisely about tools.

\subsection{What were they trying to do}

Investigate 'social solutions' that allow developers to be aware of tools. Several research has found the lack of knowledge as a factor for a task (Grossman et al 2009) \cite{grossmanSurveySoftwareLearnability2009} and in many cases, there is often a huge difference in developers knowing all the features that their IDE may offer (Campbell \& Miller 2008) \cite{campbellDesigningRefactoringTools2008} - very very true with VSCode features and my experience with people in teams, peer interaction is huge here. There are notifications or alert messages such as "tip of the day" that attempt to provide awareness, but they can be seen as a bad UX step as well (depending on how the system implements it, for eg if its an additional blocking step).

Cockburn and Williams (2000)\cite{cockburnCostsBenefitsPair2001} also reveals that many tool discovery occurs during pair programming, and this paper uses that as a research starting point.

\subsection{What did they do}

A series of interviews conducted with 18 programmers exploring how they discovered tools over the 41 different instances of peer interaction as reported, 27 were peer observation and 14 were peer recommendation.

\subsection{What did they learn}

Programmers believe that discovery from peers is effective - they discover tools from peers relatively infrequently, but this form of discovery does not occur as frequently as other modes. Another finding was that programmers can effectively discover tools from their peers both in a co-located and remote settings. Discovery from peers can be enhanced by improving programmers' ability to communicate openly and concisely about tools.

Seven discovery modes:

\begin{itemize}
    \item Peer Observation = person sees recommender, before learning
    \item Peer Recommendation = recommender sees them, and speaks
    \item Tool Encounter = exploring the tool during normal usage
    \item Tutorial = learning about something and a tool is mentioned
    \item Written Description = noticing a tool is mentioned
    \item Twitter or RSS Feed
    \item Discussion Thread = see a whole conversation about a tool
\end{itemize}
