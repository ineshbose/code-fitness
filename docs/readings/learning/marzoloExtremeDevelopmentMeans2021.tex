% Not strictly 'learning' paper

\section{“Extreme Development” as a Means for Learning Agile}

\subsection{Abstract}

During the 2020 pandemic a new modality for the capstone project in Software Engineering was introduced to our third-year students in Computer Science. They have been tasked with the development of a non trivial software product - a Twitter client capable of visual analytics - using some Agile practices, exploiting a Scrum-like process model, and using only open source tools. Due to circumstances that were either planned (in the selection of tools and requirements) or unintended (the pandemic forbade any physical meeting), the project had some interesting outcomes. The project was not easy to enact, neither for the students nor for the instructors. The main problems were two: the students were not ready to practice agile teamwork, and the open source tools they had to use were demanding and only partly suitable for the goal they were chosen for. We term this experience - where students applied an agile discipline and were required to use only open source tools - an “extreme” agile development project. This paper - written by two students together with their instructor, summarises some lessons learnt: characteristics and features of the tools and practices used, the evolution of product artifacts and some difficulties encountered, along with the solutions we adopted. An important lesson learnt is that an agile project developed by Computer Science students requires specific training in communicating correct information at the right moment, and avoiding telling “social lies” concerning the status of both the product and its development process.

\subsection{What were they trying to do}

Observe a group of students (novice developers) while they develop software (Twitter client for visual analytics) using Agile practices and open-source tools through the unexpected time of the COVID-19 Pandemic. CS Students do not like teamwork (Jackson et. al 2004).

> Students are not used to self-tracking their productivity, and even less to "team tracking", namely the act of measuring the effectiveness of their teamwork.

> In an educational setting, there is a specific problem: the students tend to develop the project as an effort necessary to pass the exam, so it is natural for them to minimize efforts and possibly lie about the real status of their process

Research questions:

\begin{itemize}
    \item Can an agile development discipline (e.g. Scrum) and open source software tools be effectively combined when training novice developers?
    \item How can we evaluate the teamwork and agility of a team of novice developers who use open source tools?
\end{itemize}

% "past research on university projects during the COVID-19 pandemic are scarce"

In this paper, we will outline the challenges one
group faced, the support given by instructors, the adaptations they implemented
and why their freedom of choice helped them learn the importance of adapting.
In order to better represent their viewpoints, some parts of this article will be
written from the point of view of the students.

\subsection{What did they do}

Ran throughout a course for third-year students.

The product to develop was a Twitter client, enriched with features for data
analytics: the product should be able to capture large sets of geolocalizable
tweets and: a) put them on a map; b) create a word cloud with their contents;
c) create a temporal diagram to show the distribution of collected tweets across
time, and so on. The main uses case were: a) using Twitter in an emergency, like
an earthquake, to collect help messages; b) using tweets to track the movements
and collect the picture of a group of travelers in a city or across a region; c)
using tweets for simple diachronic sentiment analysis.

Proposed Tools

\begin{itemize}
    \item Taiga
    \item GitLab
    \item BugZilla
    \item Mattermost
    \item SonarQube
    \item Productivity Logger
    \item Essence
\end{itemize}

Final Tool Configuration

\begin{itemize}
    \item GitLab
    \item Taiga
    \item SonarQube
    \item Telegram
    \item WakaTime
    \item Etherpad
    \item Discord
    \item MS Live Share
\end{itemize}

Evaluated quality of the final product.

\subsection{What did they learn}

Answer to research questions

\begin{itemize}
    \item We have asked our students to combine in a project of
“Extreme Development” an Agile discipline and some Open Source tools; we
believe that the combination is quite challenging and demanding for 3rd year
Computer Science students of Software Engineering. All teams completed their
projects, with a variety of grades.
    \item We have developed two quality models, one for teamwork
and one for agile maturity, which have been quite effective and useful for assessing
the results
\end{itemize}

Same output as TP3 - students learn about practices and teams.


% This paper isn't quite promising. It's more about Agile practice rather than software, and it's written by students at the University of Bologna (along with their instructor), so not a lot of cited statements, but some statements are also hitting the nail discussing students and their approach to software development (especially at uni).
