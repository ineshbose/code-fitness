\section{Perceived obstacles by novice developers adopting user interface APIs and tools}

\subsection{Abstract}

An Application-Programming Interface or API provides a set of program functions that can be used to build new applications. In this paper, we study how to use the expectation-confirmation theory (ECT) to identify API usability problems, and what obstacles a novice developer faces when learning a new API and its accompanying development tools. We conduct a study over the impact of using a visual editor on API usability and then use the expectation-confirmation theory to study perceptions about the API and the editor. We finally present a list of obstacles found in the study that can be used by others to create more usable APIs and development tools.

\subsection{What were they trying to do}

They study how to use the expectation confirmation theory (ECT) to identify API usability problems, and what obstacles a novice developer faces when learning a new API and its accompanying development tools.

\subsection{What did they do}

They conduct a study over the impact of using a visual editor on API usability and then use the expectation-confirmation theory to study perceptions about the API and the editor

The studied API is Vaadin, a Java-based open source framework for rich internet applications [1] and a visual editor for that API, the Vaadin Visual Designer

\subsection{What did they learn}

Their study showed that the context in which an API is being used is very important and should not be ignored. API developers should take the configuration of the development environment of the API in consideration when designing its tools. In addition, characteristics like quality of the API, progressive evaluation of programs written with the API and meeting user’s expectations should also be taken into account by API developers. They also discovered that if there are many learning resources that are not structured well and have key information scattered, they may end up frustrating the user in finding the right learning resource for their tasks.
