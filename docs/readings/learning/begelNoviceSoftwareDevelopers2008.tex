\section{Novice Software Developers, All Over Again}

\subsection{Abstract}

Transitions from novice to expert often cause stress and anxiety and require specialized instruction and support to enact efficiently. While many studies have looked at novice computer science students, very little research has been conducted on professional novices. We conducted a two-month in-situ qualitative case study of new software developers in their first six months working at Microsoft. We shadowed them in all aspects of their jobs: coding, debugging, designing, and engaging with their team, and analyzed the types of tasks in which they engage. We can explain many of the behaviors revealed by our analyses if viewed through the lens of newcomer socialization from the field of organizational man-agement. This new perspective also enables us to better understand how current computer science pedagogy prepares students for jobs in the software industry. We consider the implications of this data and analysis for developing new processes for learning in both university and industrial settings to help accelerate the transition from novice to expert software developer.

\subsection{What were they trying to do}

Understand how novice developers in the professional industry (different from students) progress and transition to experts
Schein \cite{gallosOrganizationDevelopmentJosseyBass2017} proposed that there were three main aspects to introducing newcomers to organizations: function, hierarchy and social networking.

In this paper, we offer information to educational researchers to support the design of computer science courses to address the anticipated needs of their students‘ transition to industrial jobs.

\subsection{What did they do}

Observing 8 new recruits at Microsoft regarding their tasks, what steps it involves and what portion of the entire task is spent performing that

\subsection{What did they learn}

Several factors influence NSDs, such as team size and mentoring in an organisation like Microsoft (parts of onboarding processes) using Peer Programming and there's no one-way, but these are traditional, common and effective methods
