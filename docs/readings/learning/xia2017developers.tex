\section{What do developers search for on the web?}

\subsection{Abstract}

Developers commonly make use of a web search engine such as Google to locate online resources to improve their productivity. A better understanding of what developers search for could help us understand their behaviors and the problems that they meet during the software development process. Unfortunately, we have a limited understanding of what developers frequently search for and of the search tasks that they often find challenging. To address this gap, we collected search queries from 60 developers, surveyed 235 software engineers from more than 21 countries across five continents. In particular, we asked our survey participants to rate the frequency and difficulty of 34 search tasks which are grouped along the following seven dimensions: general search, debugging and bug fixing, programming, third party code reuse, tools, database, and testing. We find that searching for explanations for unknown terminologies, explanations for exceptions/error messages (e.g., HTTP 404), reusable code snippets, solutions to common programming bugs, and suitable third-party libraries/services are the most frequent search tasks that developers perform, while searching for solutions to performance bugs, solutions to multi-threading bugs, public datasets to test newly developed algorithms or systems, reusable code snippets, best industrial practices, database optimization solutions, solutions to security bugs, and solutions to software configuration bugs are the most difficult search tasks that developers consider. Our study sheds light as to why practitioners often perform some of these tasks and why they find some of them to be challenging. We also discuss the implications of our findings to future research in several research areas, e.g., code search engines, domain-specific search engines, and automated generation and refinement of search queries.

\subsection{What were they trying to do}

Understand search queries made by developers for their tasks. The previous study [Bao et al. 2015a] \cite{baoTrackingAnalyzingCrossCutting2015} found that there are more than 20 queries made by developers everyday related to their development. A lot of it depends on the phase, such as requirement analysis where they search for explanation of terminologies, books and tutorials to understand business logic, or development phase where they look for third party libraries, or maintenance looking for bugs that are commonly faced hopefully with a solution provided. Understanding queries would also help understand developers' problems during the development process, to also possibly develop tools to help them (tools are available now such as VSCode extensions giving links, or GitHub autopilot).

Method of search is also investigated by Sim et al. 1998 \cite{simArchetypalSourceCode1998}, Stolee et al. 2014 \cite{stoleeSolvingSearchSource2014}, Sadowski et al. 2015 \cite{sadowskiHowDevelopersSearch2015}, Bajracharya and Lopes 2009, 2012 \cite{bajracharyaMiningSearchTopics2009,bajracharyaAnalyzingMiningCode2012} to find reasons for code searches. General search (not just dev): Broder (2002) \cite{broderTaxonomyWebSearch2002} classified queries according to their intents into navigational, informational and transactional, while Rose and Levinson (2004) \cite{roseUnderstandingUserGoals2004} classified into navigational, informational and resource. Cutrell and Guan (2007) \cite{cutrellWhatAreYou2007} made use of eye tracking technologies to explore effects of changes in presentation of search results.

\subsection{What did they do}

Investigate frequency and difficulty of searches made by developers (during development process). This was an observational study of 60 developers, interviews of 12 senior software engineers from 2 companies, and a survey of 235 software engineers from 21 countries and different backgrounds (companies, open source projects, etc). 34 search tasks were grouped into 7 categories (General Search, Debugging and Bug Fixing, Programming, Third Party Code Reuse, Tools, Database, Testing) and ranked according to frequency/difficulty.

\subsection{What did they learn}

Revealing the dimensions of search queries along with their frequencies in order to highlight the importance of

\begin{itemize}
    \item developing domain-specific search engines
    \item automated generation and refinement of search queries
    \item automated prediction of the quality of search results
    \item configuration, security and performance bug fixing
    \item knowledge sharing and community building
\end{itemize}

% (kind of like what Stack Exchange does)
