\section{What Is the Best Way For Developers to Learn New Software Tools?}

\subsection{Abstract}

The better developers can learn software tools, the faster they can start using them and the more
efficiently they can later work with them. Tutorials are supposed to help here. While in the early days of
computing, mostly text tutorials were available, nowadays software developers can choose among a huge
number of tutorials for almost any popular software tool. However, only little research was conducted to
understand how text tutorials differ from other tutorials, which tutorial types are preferred and, especially,
which tutorial types yield the best learning experience in terms of efficiency and effectiveness, especially for
programmers.
To evaluate these questions, we converted an existing video tutorial for a novel software tool into a contentequivalent text tutorial. We then conducted an experiment in three groups where 42 undergraduate students
from a software engineering course were commissioned to operate the software tool after using a tutorial:
the first group was provided only with the video tutorial, the second group only with the text tutorial and the
third group with both.
In this context, the differences in terms of efficiency were almost negligible: We could observe that participants using only the text tutorial completed the tutorial faster than the participants with the video tutorial.
However, the participants using only the video tutorial applied the learned content faster, achieving roughly
the same bottom line performance. We also found that if both tutorial types are offered, participants prefer
video tutorials for learning new content but text tutorials for looking up “missed” information.
We mainly gathered our data through questionnaires and screen recordings and analyzed it with suitable
statistical hypotheses tests. The data is available at [11].
Since producing tutorials requires effort, knowing with which type of tutorial learnability can be increased
to which extent has an immense practical relevance. We conclude that in contexts similar to ours, while it
would be ideal if software tool makers would offer both tutorial types, it seems more efficient to produce
only text tutorials instead of a passive video tutorial – provided you manage to motivate your learners to use
them

\subsection{What were they trying to do}

Trying to compare (passive) text-based tutorials with video-based tutorials and understand which is better (in terms of efficiency and learning) since producting tutorials requires effort, and knowing which is better would increase learnability and give better learning experience.

Research objective: What is the best way for developers to learn new software tools?

Research questions:

\begin{itemize}
    \item What kind of tutorial do developers prefer if both text and video tutorials are available?
    \item Which tutorial takes developers less time?
    \item Which tutorial is more effective for developers?
\end{itemize}

% every paper loves to use "a little research has been conducted [in the area we are going to talk about]"

\subsection{What did they do}

42 undergraduate software engineering students were commissioned to use a tool (an add-in for MS Excel) after going through the tutorial in the form of text (transformed from video) (group 1), video (group 2), and both (group 3). Data was gathered using questionnaires and screen recordings.

\subsection{What did they learn}

Differences in terms of efficiency were negligible and there are advantages/disadvantages to either approach (similar to Alexander 2013, DeVaney 2009, Payne et al 1992). Text tutorials were completed faster (similar to Mestre 2012), while video tutorial helped faster learning over the compartively longer period (similar to Baecker 2002, Lloyd \& Robertson (2012), van der Meij 2014). Students, however, when offered a choice, preferred using video tutorials.

% such an obvious result
