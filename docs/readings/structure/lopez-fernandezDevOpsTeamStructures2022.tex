\section{DevOps Team Structures: Characterization and Implications}

\subsection{Abstract}

Context: DevOps can be defined as a cultural movement to improve and accelerate the delivery of business value by making the collaboration between development and operations effective. Objective: This paper aims to help practitioners and researchers to better understand the organizational structure and characteristics of teams adopting DevOps. Method: We conducted an exploratory study by leveraging in-depth, semi-structured interviews with relevant stakeholders of 31 multinational software-intensive companies, together with industrial workshops and observations at organizations’ facilities that supported triangulation. We used Grounded Theory as qualitative research method to explore the structure and characteristics of teams, and statistical analysis to discover their implications in software delivery performance. Results: We describe a taxonomy of team structures that shows emerging, stable and consolidated product teams that are classified according to six variables, such as collaboration frequency, product ownership sharing, and autonomy, among others, as well as their implications on software delivery performance. These teams are often supported by horizontal teams (DevOps platform teams, Centers of Excellence, and chapters) that provide them with platform technical capabilities, mentoring and evangelization, and even temporarily may facilitate human resources. Conclusion: This study aims to strengthen evidence and support practitioners in making better informed about organizational team structures by analyzing their main characteristics and implications in software delivery performance.

\subsection{What were they trying to do}

This paper aims to help practitioners and researchers to better understand the organizational structure and characteristics of teams adopting DevOps

\subsection{What did they do}

We conducted an exploratory study by leveraging in-depth, semi-structured interviews with relevant stakeholders of 31 multinational software-intensive companies, together with industrial workshops and observations at organizations’ facilities that supported triangulation. We used Grounded Theory as qualitative research method to explore the structure and characteristics of teams, and statistical analysis to discover their implications in software delivery performance.

Research questions:

\begin{itemize}
    \item How do real-world organisations structure themselves to instilling a DevOps culture into their organisation?
    \item What implications do team structure patterns have on software delivery performance?
\end{itemize}

\subsection{What did they learn}

We describe a taxonomy of team structures that shows emerging, stable and consolidated product teams that are classified according to six variables, such as collaboration frequency, product ownership sharing, and autonomy, among others, as well as their implications on software delivery performance. These teams are often supported by horizontal teams (DevOps platform teams, Centers of Excellence, and chapters) that provide them with platform technical capabilities, mentoring and evangelization, and even temporarily may facilitate human resources.
