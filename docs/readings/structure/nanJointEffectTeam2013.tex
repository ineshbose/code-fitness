\section{Joint Effect of Team Structure and Software Architecture in Open Source Software Development}

\subsection{Abstract}

In this study, we seek to understand socio-technical interactions in a system development context via an examination of the joint effect of developer team structure and open source software (OSS) architecture on OSS development performance. Using detailed data collected from code repositories from Soure-Forge.com, we find that developer team structure and software architecture significantly moderate each other's effect on OSS development performance. Larger teams tend to produce more favorable project performance when the project being developed has a high level of structural interdependency while projects with a low level of structural interdependency require smaller teams in order to achieve better project performance. Meanwhile, centralized teams tend to have a positive impact on project performance when the OSS project has a high level of structural interdependency. However, when a project has a low level of structural interdependency, centralized teams can impair project performance. This study extends our understanding of information technology's deep engagement in organizational life and provides directions for open source practitioners to better organize their projects to achieve greater performance.

\subsection{What were they trying to do}

Understand socio-technical interactions in a system development context

\subsection{What did they do}

Examination of the joint effect of developer team structure and OSS architecture on development performance using data from source-forge.com.

\subsection{What did they learn}

Team structure and software architecture significantly moderate each other's effect on OSS development performance. Larger teams produce favorable performance when project has high level of structural interdependency. Converse for smaller teams i.e. good performance when low level of structural interdependency. That is, performance is directly proportional on size & structural interdependency.
