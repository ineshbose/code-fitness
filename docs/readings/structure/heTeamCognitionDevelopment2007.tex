\section{Team Cognition: Development and Evolution in Software Project Teams}
% possibly also team learning?

\subsection{Abstract}

In software development, team-based work structures are commonly used to accomplish complex projects. Software project teams must be able to utilize the expertise and knowledge of participants without overwhelming individual members. To efficiently leverage individuals' knowledge and expertise, software project teams develop team cognition structures that facilitate their knowledge activities. This study focuses on the emergence and evolution of team cognition in software project teams, and examines how communication activity and team diversity impact the formation of these structures. A longitudinal study was conducted of 51 database development teams. The results suggest that some forms of communication and team diversity affect the formation of team cognition. Frequency of meetings and phone calls were positively related to the formation of team cognition, while e-mail use had no effect. Gender diversity had a strong and positive effect on the development of team cognition and the effect remained stable over time. Implications for the practical potential and limitations of purposive team construction as a strategy for improving software development team performance are discussed.

\subsection{What were they trying to do}

Focus on the emergence and evolution of team cognition in software project teams, and examines how communication activity and team diversity impact the formation of these structures

> Software projects are typically complex, dynamic, and involve unstructured tasks [4, 37] \cite{brodbeckCommunicationPerformanceSoftware2001,krautCoordinationSoftwareDevelopment1995}. Execution of these projects requires knowledge and expertise from many domains [14] \cite{curtisFieldStudySoftware1988}. Teams are viewed as a primary mechanism for leveraging the specialized knowledge of individual team members [11, 39] \cite{cookeMeasuringTeamKnowledge2001,lewisMeasuringTransactiveMemory2003}. Ideally, a software project team is staffed so that both the levels and the distribution of knowledge within the team match those required for the successful completion of the project [74] \cite{walzSoftwareDesignTeam1993}. the mere presence of individuals with diverse knowledge is an insufficient condition for a software project team to achieve quality performance [17] \cite{farajCoordinatingExpertiseSoftware2000}.

\subsection{What did they do}

Conduct a longitudinal study of 51 database development teams

\subsection{What did they learn}

Affects on formation of team cognition

\begin{itemize}
    \item communication and team diversity
    \item frequency of meetings and calls (positive)
    \item email had no effect
    \item Gender diversity (strong and positive)
\end{itemize}
