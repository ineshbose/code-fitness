% also a tools paper

\section{Identifying Coordination Problems in Software Development: Finding Mismatches between Software and Project Team Structures}

\subsection{Abstract}

Today's dynamic and iterative development environment brings significant challenges for software project management. In distributed project settings, "management by walking around" is no longer an option and project managers may miss out on key project insights. The TESNA (TEchnical Social Network Analysis) method and tool aims to provide project managers both a method and a tool for gaining insights and taking corrective action. TESNA achieves this by analysing a project's evolving social and technical network structures using data from multiple sources, including CVS, email and chat repositories. Using pattern theory, TESNA helps to identify areas where the current state of the project's social and technical networks conflicts with what patterns suggest. We refer to such a conflict as a Socio-Technical Structure Clash (STSC). In this paper we report on our experience of using TESNA to identify STSCs in a corporate environment through the mining of software repositories. We find multiple instances of three STSCs (Conway's Law, Code Ownership and Project Coordination) in many of the on-going development projects, thereby validating the method and tool that we have developed.

\subsection{What were they trying to do}

Create a tool called TESNA (TEchnical Social Network Analysis) to provide project managers a method for gaining insights and taking corrective action

> While there is no single cause for the problems in Software Development, a major factor is the problem of coordinating activities while developing software systems (Kraut \& Street 1995) [1]

> Crowston (1997) [11] defines three basic types of dependencies: (i) between task and resource, (ii) between two resources and (iii) between two tasks

\subsection{What did they do}

Analyse a project's evolving social and technical network structures using data from multiple sources (including CVS, email and chat repositories). Using pattern theory, TESNA helps to identify areas where the state of project conflicts with what patterns suggest (called Socio-Technical Structure Clash STSC)

\subsection{What did they learn}

t it is difficult and labour intensive for the project manager to select appropriate patterns and keep track of their potential occurrences. Identifying STSCs can prove particularly difficult when multiple people are responsible for various tasks and when the task assignments keep changing in a dynamic and iterative software development environment. Though as team leader David suggested, the detection of a STSC does not necessarily mean that a real STSC exists.
