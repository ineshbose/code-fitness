\section{The effect of programming team structures on programming tasks}

\subsection{Abstract}

The literature recognizes two group structures for managing programming projects: Baker's chief programmer team and Weinberg's egoless team. Although each structure's success in project management can be demonstrated, this success is clearly dependent on the type of programming task undertaken. Here, for the purposes of comparison, a third project organization which lies between the other two in its communication patterns and dissemination of decision-making authority is presented. Recommendations are given for selecting one of the three team organizations depending on the task to be performed.

\subsection{What were they trying to do}

Compare two group structures for managing programming projects

\begin{itemize}
    \item Chief Programmer Team (by Baker (1972) [1] \cite{bakerChiefProgrammerTeam1972}, originally by Mills (1971) [18] \cite{mills1971chief})
    \item Egoless Programming Team (by Weinberg (1971) [28] \cite{weinbergPsychologyComputerProgramming1988})
\end{itemize}

\subsection{What did they do}

Classified performance of team based on task properties:

\begin{itemize}
    \item Difficulty
    \item Size
    \item Duration
    \item Modularity
    \item Reliability
    \item Time
    \item Sociability
\end{itemize}

Renamed team categories to names reflecting decision-making authority \& communication structure:

\begin{itemize}
    \item Democractic Decentralised (DD) i.e. egoless
    \item Controlled Decentralised (CD) i.e. a "modern" team with "no boss"; has hierarchy
    \item Controlled Centralized (CC) i.e. chief programmer
\end{itemize}

\subsection{What did they learn}

Basili and Reiter (1979) [2] \cite{basili1979investigation} found relationships between the size of a programming group and several software metrics
