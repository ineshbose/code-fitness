\section{Factors Influencing Software Development Productivity - State of the Art and Industrial Experiences}

\subsection{Abstract}

Managing software development productivity is a key issue in software organizations. Business demands for shorter time-to-market while maintaining high product quality force software organizations to
look for new strategies to increase development productivity.

Traditional, simple delivery rates employed to control hardware production processes have turned out
not to work when simply transferred to the software domain. The productivity of software production
processes may vary across development contexts dependent on numerous influencing factors. Effective
productivity management requires considering these factors. Yet, there are thousands of possible factors
and considering all of them would make no sense from the economical point of view. Therefore, productivity modeling should focus on a limited number of factors with the most significant impact on productivity.

In this chapter, we present a comprehensive overview of productivity factors recently considered by
software practitioners. The study results are based on the review of 126 publications as well as international experiences of the Fraunhofer Institute, including the most recent 13 industrial projects, 4 workshops, and 8 surveys on software productivity. The aggregated results show that the productivity of
software development processes still depends significantly on the capabilities of developers as well as
on the tools and methods they use.

\subsection{What were they trying to do}

List productivity factors considered by software practitioners

> Numerous companies already measure software productivity [102] or plan to measure it for the purpose of improving their process efficiency. many software organizations actually still use the simplified definition of productivity
provided in [65]

> productivity of industrial production processes has been measured as the ratio of units of output divided by units of input [95]. This perspective was transferred into the software development context and is usually defined as productivity [67] or efficiency [123].

> software production processes seem to be significantly more difficult than production processes in other industries [1][8][28]. This is mainly because software organizations typically develop new products as opposed to fabricating the same product over and over again % not happy with this statement

\subsection{What did they do}

Conducted literature review. Results based on the review of 126 publications as well as international experiences of the Fraunhofer Institute including recent 13 industrial projects, 4 workshops and 8 surveys on software productivity

\subsection{What did they learn}

Productivity of software development processes still depends significantly on the capabilities of developers as well as on the tools and methods they use

> Selecting the right factors is just a first step towards quantitative productivity management
