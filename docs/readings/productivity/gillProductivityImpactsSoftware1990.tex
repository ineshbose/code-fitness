\section{Productivity impacts of software complexity and developer experience}

\subsection{Abstract}

The high costs of developing and maintaining software have become widely recognized as major
obstacles to the continued successful use of information technology. Unfortunately, against this backdrop
of high and rising costs stands only a hmited amount of research in measuring software development
productivity and m understanding the effect of software complexity on variations in productivity. The
current research is an in-depth study of a number of software development projects undertaken by an
aerospace defense contracting firm in the late 1980s. These data allow for the analysis of the impact
of software complexity and of varying levels of staff expertise on both new development and
maintenance.
In terms of the main research questions, the data support the notion that more complex
systems, as measured by size-adjusted McCabe cyclomatic complexity, require more effort to maintain.
For this data set, the Myers and Hansen variations were found to be essentially equivalent to the
original McCabe specification. Software engineers with more experience were more productive, but this labor production input factor revealed declining marginal returns to experience. The average net marginal impact of additional experience (minus its higher cost) was positive.

\subsection{What were they trying to do}

Questions:

\begin{itemize}
    \item Are more experienced programmers more productive, and, if so, by how much?
    \item How does the proportion of time spent in the design phase affect the productivity?
\end{itemize}

> \$140 billion is spent annually worldwide on software [Boehm, 1987].
> software development productivity has been measured as a simple output/input ratio, most typically source lines of code (SLOC) per work-month [Boehm, 1981, Conte, et al., 1986, Jones, 1986].

\subsection{What did they do}

in-depth study of a number of software development projects undertaken by an aerospace defense contracting firm in the late 1980s.

\subsection{What did they learn}

Software engineers with more experience were more productive, but this labor production input factor revealed declining marginal returns to experience. Greater emphasis on the design phase is associated with increased productivity.
