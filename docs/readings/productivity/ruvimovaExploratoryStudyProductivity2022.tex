% also related to structure

\section{An Exploratory Study of Productivity Perceptions in Software Teams}

\subsection{Abstract}

Software development is a collaborative process requiring a careful balance of focused individual
effort and team coordination. Though questions of individual productivity have been widely examined in
past literature, less is known about the interplay between developers’ perceptions of their own productivity
as opposed to their team’s. In this paper, we present an analysis of 624 daily surveys and 2899 self-reports
from 25 individuals across five software teams in North America and Europe, collected over the course
of three months. We found that developers tend to operate in fluid team constructs, which impacts
team awareness and complicates gauging team productivity. We also found that perceived individual
productivity most strongly predicted perceived team productivity, even more than the amount of team
interactions, unplanned work, and time spent in meetings. Future research should explore how fluid team
structures impact individual and organizational productivity.

\subsection{What were they trying to do}

Measure productivity as it is a subjective metric and depends on individuals and teams

Recent research has placed more
emphasis on the individuals involved in the development process, and in particular, on how they view their own productivity [Amy Ko. 2019. Why We Should Not Measure Productivity, (André Meyer, Thomas Fritz, Gail Murphy, and Thomas Zimmermann. 2014. Software Developers’ Perceptions of Productivity)]. Recent research also suggests that the measurement of productivity needs to include multiple dimensions of individual metrics, including such aspects as satisfaction, activity, and flow [Young-Ho Kim, Eun Kyoung Choe, Bongshin Lee, and Jinwook Seo. 2019. Understanding Personal Productivity: How Knowledge Workers Define, Evaluate, and Reflect on Their Productivity, (Gloria Mark, Shamsi T. Iqbal, Mary Czerwinski, Paul Johns, and Akane Sano. 2016)].
Lakhanpa et al. (1993) using a one-time
survey, found team cohesion to be the dominating factor when
investigating team performance.

> productivity is defined as the ratio of outputs over
inputs, where output is value produced and input is time or other
costs invested (Pritchard 1995)

Quantifying Software Development Productivity

Automated process metrics:

\begin{itemize}
    \item Lines of source code (SLOC) written [(Prem Devanbu, Sakke Karstu, Walcélio Melo, and William Thomas. 1996), (C. E. Walston and C. P. Felix. 1977)]
    \item function points [1 - Allan J. Albrecht. 1979, 27 - C. Jones. 1994]
    \item changes request fulfilled [14 - Marcelo Cataldo, James D. Herbsleb, and Kathleen M. Carley. 2008, 42 - Courtney Miller, Paige Rodeghero, Margaret-Anne Storey, Denae Ford, and Thomas Zimmermann. 2021. "How Was Your Weekend?"]
    \item Drawback - they are highly objective and may only capture small part of developer's work
\end{itemize}

This paper uses self-reported perceived productivity.

Team productivity research:

\begin{itemize}
    \item  Boehm (1987) indicated that productivity on a software development project is most affected by who develops the system and how well they are organized and managed as a team
    \item Bendifallah and Scacchi (1989) found explanation in terms of recurring teamwork structures, categorised by patterns of interaction
    \item Lakhanpa et al. (1993) found team cohesion to be the dominating factor
    \item Edna Dias Canedo and Giovanni Almeida Santos. 2019 [12] revealed 37 factors to be pertinent to team productivity including individual characteristics such as work experience [21 - Claudia De O. Melo, Daniela S. Cruzes, Fabio Kon, and Reidar Conradi. 2013] and skill/competence [36 - Katrina D. Maxwell and Pekka Forselius. 2000., 45 - Edson Oliveira, Tayana Conte, Marco Cristo, and Emilia Mendes. 2016.]
    \item collaboration [16 - Victor Clincy. 2003. Software Development Productivity, 17 - Broderick Crawford, Ricardo Soto, Claudio León de la Barra, Kathleen Crawford, and Eduardo Olguín. 2014]
    \item team member availability [21 - Claudia De O. Melo, Daniela S. Cruzes, Fabio Kon, and Reidar Conradi. 2013, 36 - Katrina D. Maxwell and Pekka Forselius. 2000]
    \item ease of communication [54 - S. Wagner and M. Ruhe. 2018., 56 - Murat Yilmaz, Rory O’Connor, and Paul Clarke. 2016. Effective Social Productivity]
    \item Ko [30 Amy Ko. 2019. Individual, Team, Organization, and Market: Four Lense] notes four lenses through which to think: individual, team, market and full-spectrum
\end{itemize}

> Job aspects that motivate developers or bring enjoyment correlate with higher productivity and job satisfaction [5 Sarah Beecham, Nathan Baddoo, Tracy Hall, Hugh Robinson, and Helen Sharp. 2008. Motivation in Software Engineering: A systematic literature review, 29 - Young-Ho Kim, Eun Kyoung Choe, Bongshin Lee, and Jinwook Seo. 2019.]. Positive affect states, indicating happier moods, have been shown to boost developer problem-solving skills and productivity [2 Teresa M. Amabile and Steven J. Kramer. 2011. The Progress Principle, 25 Daniel Graziotin, Xiaofeng Wang, and Pekka Abrahamsson. 2014, 40 André Meyer, Thomas Fritz, Gail Murphy, and Thomas Zimmermann. 2014, 44 S. C. Müller and T. Fritz. 2015. Stuck and Frustrated or in Flow and Happy]. Likewise, developer’s moods were shown to influence performance on certain tasks, such as debugging [28 I. Khan, W. Brinkman, and R. Hierons. 2010. Do moods affect programmers].

Research questions

\begin{itemize}
    \item What is the relationship between perceived team and individual productivity in software teams?
    \item Which team factors affect individual and team productivity perceptions?
\end{itemize}

\subsection{What did they do}

Conducted interviews and surveys regularly with 25 individuals over three months

\subsection{What did they learn}

Realised that teams are highly fluid varying in size, members, structure & organisation causing different aspects/dependencies to productivity
