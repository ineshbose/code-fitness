\section{An Exploratory Study of Productivity Perceptions in Software Teams}
% also related to structure

\subsection{Abstract}

Software development is a collaborative process requiring a careful balance of focused individual effort and team coordination. Though questions of individual productivity have been widely examined in past literature, less is known about the interplay between developers’ perceptions of their own productivity as opposed to their team’s. In this paper, we present an analysis of 624 daily surveys and 2899 self-reports from 25 individuals across five software teams in North America and Europe, collected over the course of three months. We found that developers tend to operate in fluid team constructs, which impacts team awareness and complicates gauging team productivity. We also found that perceived individual productivity most strongly predicted perceived team productivity, even more than the amount of team interactions, unplanned work, and time spent in meetings. Future research should explore how fluid team structures impact individual and organizational productivity.

\subsection{What were they trying to do}

Measure productivity as it is a subjective metric and depends on individuals and teams

Recent research has placed more emphasis on the individuals involved in the development process, and in particular, on how they view their own productivity [\cite{koWhyWeShould2019}, \cite{meyerSoftwareDevelopersPerceptions2014}]. Recent research also suggests that the measurement of productivity needs to include multiple dimensions of individual metrics, including such aspects as satisfaction, activity, and flow [\cite{kimUnderstandingPersonalProductivity2019}, \cite{markNeuroticsCanFocus2016}]. Lakhanpa et al. (1993) \cite{lakhanpalUnderstandingFactorsInfluencing1993} using a one-time survey, found team cohesion to be the dominating factor when investigating team performance.

> productivity is defined as the ratio of outputs over inputs, where output is value produced and input is time or other costs invested (Pritchard 1995) \cite{pritchard1995productivity}

Quantifying Software Development Productivity

Automated process metrics:

\begin{itemize}
    \item Lines of source code (SLOC) written [\cite{devanbuAnalyticalEmpiricalEvaluation1996}, \cite{walstonMethodProgrammingMeasurement1977}]
    \item function points [\cite{albrecht1979measuring}, \cite{computerstaffSoftwareMetricsGood1994}]
    \item changes request fulfilled [\cite{cataldoSociotechnicalCongruenceFramework2008}, \cite{millerHowWasYour2021}]
    \item Drawback - they are highly objective and may only capture small part of developer's work
\end{itemize}

This paper uses self-reported perceived productivity.

Team productivity research:

\begin{itemize}
    \item  Boehm (1987) \cite{boehmImprovingSoftwareProductivity1987} indicated that productivity on a software development project is most affected by who develops the system and how well they are organized and managed as a team
    \item Bendifallah and Scacchi (1989) \cite{bendifallahWorkStructuresShifts1989} found explanation in terms of recurring teamwork structures, categorised by patterns of interaction
    \item Lakhanpa et al. (1993) \cite{lakhanpalUnderstandingFactorsInfluencing1993} found team cohesion to be the dominating factor
    \item Edna Dias Canedo and Giovanni Almeida Santos. 2019 [12] \cite{canedoFactorsAffectingSoftware2019} revealed 37 factors to be pertinent to team productivity including individual characteristics such as work experience \cite{deo.meloInterpretativeCaseStudies2013} and skill/competence [\cite{maxwellBenchmarkingSoftwareDevelopmentProductivity2000}, \cite{oliveiraSoftwareProjectManagers2016}]
    \item collaboration [\cite{clincySoftwareDevelopmentProductivity2003}, \cite{stephanidisHCIInternational20152015}]
    \item team member availability [\cite{meloInterpretativeCaseStudies2013}, \cite{maxwellBenchmarkingSoftwareDevelopmentProductivity2000}]
    \item ease of communication [\cite{wagnerSystematicReviewProductivity2018}, \cite{yilmazEffectiveSocialProductivity2016}]
    \item Ko [\cite{koIndividualTeamOrganization2019}] notes four lenses through which to think: individual, team, market and full-spectrum
\end{itemize}

> Job aspects that motivate developers or bring enjoyment correlate with higher productivity and job satisfaction [\cite{beechamMotivationSoftwareEngineering2008}, \cite{kimUnderstandingPersonalProductivity2019}]. Positive affect states, indicating happier moods, have been shown to boost developer problem-solving skills and productivity [\cite{amabileProgressPrincipleUsing2011}, \cite{graziotinFeelingsMatterCorrelation2015}, \cite{meyerSoftwareDevelopersPerceptions2014}, \cite{mullerStuckFrustratedFlow2015}]. Likewise, developer’s moods were shown to influence performance on certain tasks, such as debugging [\cite{khanMoodsAffectProgrammers2011}].

Research questions

\begin{itemize}
    \item What is the relationship between perceived team and individual productivity in software teams?
    \item Which team factors affect individual and team productivity perceptions?
\end{itemize}

\subsection{What did they do}

Conducted interviews and surveys regularly with 25 individuals over three months

\subsection{What did they learn}

Realised that teams are highly fluid varying in size, members, structure & organisation causing different aspects/dependencies to productivity
