\section{Developer experience: Concept and definition}

\subsection{Abstract}

New ways of working such as globally distributed development or the integration of self-motivated external developers into software ecosystems will require a better and more comprehensive understanding of developers' feelings, perceptions, motivations and identification with their tasks in their respective project environments. User experience is a concept that captures how persons feel about products, systems and services. It evolved from disciplines such as interaction design and usability to a much richer scope that includes feelings, motivations, and satisfaction. Similarly, developer experience could be defined as a means for capturing how developers think and feel about their activities within their working environments, with the assumption that an improvement of the developer experience has positive impacts on characteristics such as sustained team and project performance. This article motivates the importance of developer experience, sketches related approaches from other domains, proposes a definition of developer experience that is derived from similar concepts in other domains, describes an ongoing empirical study to better understand developer experience, and finally gives an outlook on planned future research activities.

\subsection{What were they trying to do}

motivates the importance of developer experience, sketches related approaches from other domains

> Software development is an inherently human-based, intellectual activity [1 \cite{endresHandbookSoftwareSystems2003}]. Many studies show that human factors are the most important factors for software development in many different development environments, both in terms of performance [2 \cite{sackmanExploratoryExperimentalStudies1968}], [3 \cite{demarcoProgrammerPerformanceEffects1985}] and quality [4 \cite{mockusOrganizationalVolatilityIts2010}], [5 \cite{nagappanInfluenceOrganizationalStructure2008}], [6 \cite{birdPuttingItAll2009}], [7 \cite{trendowiczChapterFactorsInfluencing2009}].

> Research on GSD has shown
that lack of trust, difficulties in communication, and lack
of identification with project goals negatively impact the
success of projects [11 \cite{smiteEmpiricalEvidenceGlobal2010}], [13 \cite{holmstromAgilePracticesReduce2006}], [14 \cite{hymanCreativeChaosHighperformance1993}]

> a multitude of factors influence developer productivity (Nelson-Jones law).
Summarizing research on productivity factors, Endres et al.
come up with the conclusion that the number of factors
ranges in the hundreds or thousands [1 \cite{endresHandbookSoftwareSystems2003}]. In addition, as
DeMarco and Lister point out, there are hundreds of ways to
influence productivity, and most of them are non-technical
[3 \cite{demarcoProgrammerPerformanceEffects1985}]. Overemphasis on productivity is the best way to lose it [1 \cite{endresHandbookSoftwareSystems2003}].

> The word “developer” refers here to anyone who is engaged in the activity of developing software, and “experience” refers here to involvement, not to being experienced, although the two are interlinked.

Question: How do practitioners characterize a
software developer’s experience in a specific development
environment and what kind of impact factors on this experience do they consider as relevant in this respective
environment?

\subsection{What did they do}

proposes a definition of developer experience that is derived from similar concepts in other domains
